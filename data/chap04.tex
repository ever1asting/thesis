\chapter{实验结果}
\label{cha:result}

\section{子节点和剪枝算法实验结果}

首先给出在park\_joy\_90p\_8\_420这个视频上的实验结果。在使用基于子节点和剪枝的早终止算法前后,
视频编码在比特率、画质、编码时间三方面的变化如 \ref{tab:result-sum-park-bitrate}、\ref{tab:result-sum-park-psnr}、
\ref{tab:result-sum-park-time} 所示。

\begin{table}[H]
  \centering
    \caption{park\_joy实验,比特率对比}
    \label{tab:result-sum-park-bitrate}
    \begin{tabularx}{\linewidth}{XXXX}
      \toprule[1.5pt]
      预设比特率 & 原始比特率 & 改进算法比特率 & 比较 \\
      \midrule[1pt]
      200 & 1 & 2 & 3  \\
      250 & 7 & 4 & 10 \\
      300 & 1 & 2 & 3  \\
      350 & 7 & 4 & 10 \\
      400 & 1 & 2 & 3  \\
      450 & 7 & 4 & 10 \\
      500 & 1 & 2 & 3  \\
      \bottomrule[1.5pt]
    \end{tabularx}
\end{table}

\begin{table}[H]
  \centering
    \caption{park\_joy实验,PSNR对比}
    \label{tab:result-sum-park-psnr}
    \begin{tabularx}{\linewidth}{XXXX}
      \toprule[1.5pt]
      预设比特率 & 原始PSNR & 改进算法PSNR & 比较 \\
      \midrule[1pt]
      200 & 1 & 2 & 3  \\
      250 & 7 & 4 & 10 \\
      300 & 1 & 2 & 3  \\
      350 & 7 & 4 & 10 \\
      400 & 1 & 2 & 3  \\
      450 & 7 & 4 & 10 \\
      500 & 1 & 2 & 3  \\
      \bottomrule[1.5pt]
    \end{tabularx}
\end{table}

\begin{table}[H]
  \centering
    \caption{park\_joy实验,编码用时对比}
    \label{tab:result-sum-park-time}
    \begin{tabularx}{\linewidth}{XXXX}
      \toprule[1.5pt]
      预设比特率 & 原始时间 & 改进算法时间 & 比较 \\
      \midrule[1pt]
      200 & 1 & 2 & 3  \\
      250 & 7 & 4 & 10 \\
      300 & 1 & 2 & 3  \\
      350 & 7 & 4 & 10 \\
      400 & 1 & 2 & 3  \\
      450 & 7 & 4 & 10 \\
      500 & 1 & 2 & 3  \\
      \bottomrule[1.5pt]
    \end{tabularx}
\end{table}

啊啊啊ss1



\section{加权子节点和剪枝算法实验结果}

与基于子节点和剪枝的早终止算法实验相同,首先给出在park\_joy\_90p\_8\_420这个视频上的实验结果,在比特率、
画质、编码时间三方面的对比如 \ref{tab:result-weighted-park-bitrate}、\ref{tab:result-weighted-park-psnr}、
\ref{tab:result-weighted-park-time} 所示。


\begin{table}[H]
  \caption{park\_joy实验,比特率对比}
    \label{tab:result-weighted-park-bitrate}
    \begin{tabularx}{\linewidth}{XXXX}
      \toprule[1.5pt]
      预设比特率 & 原始比特率 & 改进算法比特率 & 比较 \\
      \midrule[1pt]
      200 & 1 & 2 & 3  \\
      250 & 7 & 4 & 10 \\
      300 & 1 & 2 & 3  \\
      350 & 7 & 4 & 10 \\
      400 & 1 & 2 & 3  \\
      450 & 7 & 4 & 10 \\
      500 & 1 & 2 & 3  \\
      \bottomrule[1.5pt]
    \end{tabularx}
\end{table}

\begin{table}[H]
  \centering
    \caption{park\_joy实验,PSNR对比}
    \label{tab:result-weighted-park-psnr}
    \begin{tabularx}{\linewidth}{XXXX}
      \toprule[1.5pt]
      预设比特率 & 原始PSNR & 改进算法PSNR & 比较 \\
      \midrule[1pt]
      200 & 1 & 2 & 3  \\
      250 & 7 & 4 & 10 \\
      300 & 1 & 2 & 3  \\
      350 & 7 & 4 & 10 \\
      400 & 1 & 2 & 3  \\
      450 & 7 & 4 & 10 \\
      500 & 1 & 2 & 3  \\
      \bottomrule[1.5pt]
    \end{tabularx}
\end{table}

\begin{table}[H]
  \centering
    \caption{park\_joy实验,编码用时对比}
    \label{tab:result-weighted-park-time}
    \begin{tabularx}{\linewidth}{XXXX}
      \toprule[1.5pt]
      预设比特率 & 原始时间 & 改进算法时间 & 比较 \\
      \midrule[1pt]
      200 & 1 & 2 & 3  \\
      250 & 7 & 4 & 10 \\
      300 & 1 & 2 & 3  \\
      350 & 7 & 4 & 10 \\
      400 & 1 & 2 & 3  \\
      450 & 7 & 4 & 10 \\
      500 & 1 & 2 & 3  \\
      \bottomrule[1.5pt]
    \end{tabularx}
\end{table}


\section{阈值剪枝算法实验结果}

\section{小结}