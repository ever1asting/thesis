\thusetup{
  %******************************
  % 注意:
  %   1. 配置里面不要出现空行
  %   2. 不需要的配置信息可以删除
  %******************************
  %
  %=====
  % 秘级
  %=====
  secretlevel={秘密},
  secretyear={10},
  %
  %=========
  % 中文信息
  %=========
  ctitle={基于早终止的AV1视频编码加速},
  cdegree={工学硕士},
  cdepartment={计算机科学与技术系},
  cmajor={计算机科学与技术},
  cauthor={杨皖宁},
  csupervisor={温江涛教授},
  % cassosupervisor={陈文光教授}, % 副指导老师
  % ccosupervisor={某某某教授}, % 联合指导老师
  % 日期自动使用当前时间,若需指定按如下方式修改:
  % cdate={超新星纪元},
  %
  % 博士后专有部分
  cfirstdiscipline={计算机科学与技术},
  cseconddiscipline={系统结构},
  postdoctordate={2009年7月——2011年7月},
  id={编号}, % 可以留空: id={},
  udc={UDC}, % 可以留空
  catalognumber={分类号}, % 可以留空
  %
  %=========
  % 英文信息
  %=========
  etitle={An Introduction to \LaTeX{} Thesis Template of Tsinghua University v\version},
  % 这块比较复杂,需要分情况讨论:
  % 1. 学术型硕士
  %    edegree:必须为Master of Arts或Master of Science(注意大小写)
  %             “哲学、文学、历史学、法学、教育学、艺术学门类,公共管理学科
  %              填写Master of Arts,其它填写Master of Science”
  %    emajor:“获得一级学科授权的学科填写一级学科名称,其它填写二级学科名称”
  % 2. 专业型硕士
  %    edegree:“填写专业学位英文名称全称”
  %    emajor:“工程硕士填写工程领域,其它专业学位不填写此项”
  % 3. 学术型博士
  %    edegree:Doctor of Philosophy(注意大小写)
  %    emajor:“获得一级学科授权的学科填写一级学科名称,其它填写二级学科名称”
  % 4. 专业型博士
  %    edegree:“填写专业学位英文名称全称”
  %    emajor:不填写此项
  edegree={Doctor of Engineering},
  emajor={Computer Science and Technology},
  eauthor={Xue Ruini},
  esupervisor={Professor Zheng Weimin},
  eassosupervisor={Chen Wenguang},
  % 日期自动生成,若需指定按如下方式修改:
  % edate={December, 2005}
  %
  % 关键词用“英文逗号”分割
  ckeywords={AV1, 早终止, 视频编码, 树剪枝},
  ekeywords={AV1, early-termination, video coding, tree pruning}
}

% 定义中英文摘要和关键字
\begin{cabstract}
  随数字媒体娱乐的发展,人们对视频的要求愈发增加,更高的画质、更小的体积是多种多样需求的基础,
  而这些都离不开视频编码技术。为解决主流视频编码标准的高昂版权税所带来的困扰,Google于2010年
  启动WebM计划,旨在定制开源、免版权税的视频编码标准,历经VP8、VP9两代编码,新一代的视频编码
  标准AV1于2015年开始制定,并于2017年底完成了基本框架的编写。

  对于新生的视频编码标准AV1,其在编码速度方面不尽如人意,通过移植在已有视频编码标准上的加速算法
  来加速,是有广阔的空间与前景的。本文着眼于基于早终止思想的视频编码加速算法,以剪枝策略为核心,
  通过不同的剪枝条件移植了几种加速算法。移植的算法是基于K Choi和SH Park等人的工作 \cite{choi2011coding},
  笔者根据AV1标准的实际情况进行了一定程度的移植、修改与改进。

  本文首先介绍了VP9标准下,视频编码的主要流程,然后通过对比介绍了AV1标准的改动点;后续有简入难,
  介绍了三种早终止加速算法,并在随后给出了定量化的实验结果;文末对于所做的工作做了总结,并对
  未来的工作作出展望。

\end{cabstract}

% 如果习惯关键字跟在摘要文字后面,可以用直接命令来设置,如下:
% \ckeywords{\TeX, \LaTeX, CJK, 模板, 论文}

\begin{eabstract}
  With the development of digital media entertainment, people's demand for video is increasing. 
  Higher quality and smaller volume are the foundation of requirements. And these can not be 
  separated from the video coding technology. To solve the high copyright tax problem of mainstream 
  video coding standards, Google launched the WebM program in 2010 to customize open source, royalty 
  free video coding standards. After the two standards VP8 and VP9, the new generation of video coding 
  standard AV1 was formulated in 2015, and the basic framework was completed at the end of 2017.

  The new video coding standard AV1, which is unsatisfactory in the coding speed, has a broad space 
  and prospect through the acceleration algorithm transplanted on the existing video coding standard.
  This paper focuses on the accelerating algorithm of video coding based on the idea of early termination. 
  With the pruning strategy as the core, several accelerating algorithms are transplanted through 
  different pruning conditions.The transplant algorithm is based on the work of K Choi and 
  SH Park et al \cite{choi2011coding}. I have carried out a certain degree of transplant, 
  modification and improvement according to the actual situation of the AV1 standard.
  
  This paper first introduces the main process of video coding under the VP9 standard, and then 
  introduces the modifications of AV1 standard through comparison. Three kinds of early termination 
  acceleration algorithms are introduced, and experimental results are given afterwards. 
  At the end of the paper, the work done is summarized, and the future work is prospected.

\end{eabstract}

% \ekeywords{\TeX, \LaTeX, CJK, template, thesis}
