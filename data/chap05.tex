\chapter{结论}
\label{cha:conclusion}

\section{工作总结}

随着人们对数字媒体娱乐要求的提高,传统视频编码标准将难以处理高分辨率的视频。而正处在定制中的
AV1视频编码标准一方面有望适应更高分辨率的4K等高清视频,实现编码时间、压缩率等方面的提升;另一方面,
作为VP9的下一代编码标准,其作为新一代免版权税开源视频编码标准的代表,将有力地推动视频编码相关
的研究及应用。

作为最新一代的视频编码标准,AV1距离实际使用仍有一段距离,现有实现框架在性能上难以令人满意。本文
正是以对AV1标准的实现进行优化、缩短编码时间为目的,着眼于已有视频编码标准上的早终止算法进行移植,
通过移植、简化、改进等方法提出并实现了三种剪枝早终止策略来加速。基于子节点和进行剪枝的早终止加速算法
简明而有效,在保持比特率和视频质量几乎不变的前提下,将编码时间缩短了15\%左右。在此基础上进行改进,
针对AV1中四种“不规则”的划分方法提出了基于子节点加权和进行剪枝的早终止算法,将编码时间进一步缩短,
达到20\%左右。基于阈值剪枝的早终止加速算法受限于实验时间有限,无法给出一个普遍使用的阈值,但在
基于一些统计结果提出的几个阈值,在小范围的测试视频上也取得了不错的效果。

\section{进一步的工作}

本文所提出的算法虽然取得了一些成果,但受限于时间因素,一方面实现的算法仍有较大的改进余地,另一方面
很多从其他角度出的早终止算法也有移植的可能。笔者虽然无法实现这些算法,但仍可以做一些展望:

\begin{itemize}
    \item 完善基于阈值剪枝的早终止加速算法。对更多的视频进行分析,提出一个更为合理、普遍适用的阈值,
    或是总结出一套计算阈值的方法,都是可行的改进思路。
    \item 使用运动向量的早终止加速算法。可以参考的工作为Libo Yang等人提出的方法\cite{yang2005effective}。
    根据运动矢量,尤其是零运动矢量进行早终止优化,是一种可行性很高的方案。
    \item 使用SVM对节点的划分方式进行预测的早终止加速算法。已有的工作有Shen Xiaolin 等人的成果可以
    参考\cite{shen2013cu}。由于SVM本身属于较为经典且成熟的算法,因此该方法的可移植性也相对不错。而
    难点在于一方面特征空间的选取上可能性较多,由此带来很多的不确定性;而另一方面则是AV1中的划分方式
    有九种,分类的增多意味着移植算法的效果可能会下降。
    \item 针对T型划分做优化。AV1中最具体创新性的就是子节点的四种T型划分方式,针对这四种划分方式做
    优化,可能是使AV1标准胜过其他已有视频编码标准的一个突破点。
\end{itemize}

本文提出的算法不尽完美,希望对于后续的相关工作起到一定的铺垫作用,也希望针对AV1标准的优化早日达到一个
令人满意的程度,进而投入到实际应用中。