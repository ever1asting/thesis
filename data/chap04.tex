\chapter{实验结果}
\label{cha:result}

\section{子节点和剪枝算法实验结果}

首先给出在park\_joy视频上的实验结果。在使用基于子节点和剪枝的早终止算法前后,
考察视频编码在比特率、画质、编码时间三方面的变化,以下标PRO指代笔者所提出的算法、而REF指代不做改进时的情况,
比特率的变化以 $\Delta Bitrate[ (B_{PRO} - B_{REF})/B_{REF} * 100 ]$ 来衡量;画质的变化则用PSNR度量,
比较时使用 $ \Delta PSNR(P_{PRO} - P_{REF})$ 作比较;而编码时间的比较也使用相对减少的百分比
 $ \Delta Time[ (T_{PRO} - T_{REF})/T_{REF} * 100 ]$ 衡量。具体结果如 表~\ref{tab:result-sum-park-bitrate}、
表~\ref{tab:result-sum-park-psnr}、 表~\ref{tab:result-sum-park-time} 所示,需要说明的是,对单一视频
测试时,需要预先设定期望的比特率,表格中以 $Bitrate_{cond}$ 标识。

\begin{table}[H]
  \centering
    \caption{子节点和剪枝算法park\_joy实验(比特率)}
    \label{tab:result-sum-park-bitrate}
    \begin{tabularx}{\linewidth}{XXXX}
      \toprule[1.5pt]
      $Bitrate_{cond}$(kb/s) & $Bitrate_{REF}$(b/s) & $Bitrate_{PRO}$(b/s) & $\Delta Bitrate$(\%) \\
      \midrule[1pt]
      200 & 297160 & 294880 & -0.77  \\
      250 & 321160 & 319600 & -0.49 \\
      300 & 344760 & 342120 & -0.77  \\
      350 & 368800 & 365560 & -0.88 \\
      400 & 392200 & 391320 & -0.22  \\
      450 & 416800 & 413520 & -0.79 \\
      500 & 437720 & 436760 & -0.22  \\
      \bottomrule[1.5pt]
    \end{tabularx}
\end{table}

\begin{table}[H]
  \centering
    \caption{子节点和剪枝算法park\_joy实验(PSNR)}
    \label{tab:result-sum-park-psnr}
    \begin{tabularx}{\linewidth}{XXXX}
      \toprule[1.5pt]
      $Bitrate_{cond}$(kb/s) & $PSNR_{REF}$(dB) & $PSNR_{PRO}$(dB) & $\Delta PSNR$(dB) \\
      \midrule[1pt]
      200 & 34.576 & 34.809 & 0.23  \\
      250 & 35.573 & 35.780 & 0.21 \\
      300 & 36.420 & 36.591 & 0.17  \\
      350 & 37.135 & 37.250 & 0.12 \\
      400 & 37.755 & 37.936 & 0.18  \\
      450 & 38.305 & 38.464 & 0.16 \\
      500 & 38.861 & 39.073 & 0.21  \\
      \bottomrule[1.5pt]
    \end{tabularx}
\end{table}

\begin{table}[H]
  \centering
    \caption{子节点和剪枝算法park\_joy实验(编码时间)}
    \label{tab:result-sum-park-time}
    \begin{tabularx}{\linewidth}{XXXX}
      \toprule[1.5pt]
      $Bitrate_{cond}$(kb/s) & $Time_{REF}$(ms) & $Time_{PRO}$(ms) & $\Delta Time$(\%) \\
      \midrule[1pt]
      200 & 492815 & 445677 & -9.56  \\
      250 & 545196 & 423004 & -22.41 \\
      300 & 541272 & 450985 & -16.68  \\
      350 & 560691 & 481394 & -14.14 \\
      400 & 573799 & 479245 & -16.48  \\
      450 & 540554 & 490609 & -9.24 \\
      500 & 597913 & 509196 & -14.84  \\
      \bottomrule[1.5pt]
    \end{tabularx}
\end{table}


可以看到,针对park\_joy视频,基于子节点和剪枝的早终止加速算法有着不错的效果:在
PSNR与比特率几乎没有变化的情况下,所需要的编码时间缩减幅度最小有9.24\%,最大有22.41\%,平均来看,
这一算法能够使编码时间缩短14.76\%。此外,笔者还在miss\_am视频上做了相同的实验,结果中,比特率
和PSNR几乎不变,而编码时间的变化如 表~\ref{tab:result-sum-miss-time} 所示。

% TODO: miss_am_qcif结果
\begin{table}[H]
  \centering
    \caption{子节点和剪枝算法miss\_am\_qcif实验(编码时间)}
    \label{tab:result-sum-miss-time}
    \begin{tabularx}{\linewidth}{XXXX}
      \toprule[1.5pt]
      $Bitrate_{cond}$(kb/s) & $Time_{REF}$(ms) & $Time_{PRO}$(ms) & $\Delta Time$(\%) \\
      \midrule[1pt]
      200 & 35468689 & 31944684 & -9.94 \\
      250 & 37811001 & 33324426 & -11.87 \\
      300 & 39371983 & 36279519 & -7.85 \\
      350 & 44189312 & 42550631 & -3.71 \\
      400 & 49631606 & 44715847 & -9.90 \\
      \bottomrule[1.5pt]
    \end{tabularx}
\end{table}

可以看到在体积更大的视频miss\_am上测试时,基于子节点和剪枝的早终止算法产生了最差3.71\%最好11.87\%的
编码时间缩短,平均来看该算法减少8.65\%的编码时间。编码时间的缩短幅度较park\_joy上的更小,可能的原因是随着视频的
增大,编码树的结构更为复杂,较深的节点上的剪枝带来的收益变小所致。


\section{加权子节点和剪枝算法实验结果}

与基于子节点和剪枝的早终止算法实验相同,首先给出在park\_joy这个视频上的实验结果,在比特率、
画质、编码时间三方面的对比如 表~\ref{tab:result-weighted-park-bitrate}、表~\ref{tab:result-weighted-park-psnr}、
表~\ref{tab:result-weighted-park-time} 所示。

\begin{table}[H]
  \caption{子节点加权和剪枝算法park\_joy实验(比特率)}
    \label{tab:result-weighted-park-bitrate}
    \begin{tabularx}{\linewidth}{XXXX}
      \toprule[1.5pt]
      $Bitrate_{cond}$(kb/s) & $Bitrate_{REF}$(b/s) & $Bitrate_{PRO}$(b/s) & $\Delta Bitrate$(\%) \\
      \midrule[1pt]
      200 & 297160 & 295760 & -0.47  \\
      250 & 321160 & 319480 & -0.52 \\
      300 & 344760 & 343000 & -0.51  \\
      350 & 368800 & 365680 & -0.85 \\
      400 & 392200 & 390760 & -0.37  \\
      450 & 416800 & 412600 & -1.00 \\
      500 & 437720 & 435640 & -0.48  \\
      \bottomrule[1.5pt]
    \end{tabularx}
\end{table}

\begin{table}[H]
  \centering
    \caption{子节点加权和剪枝算法park\_joy实验(PSNR)}
    \label{tab:result-weighted-park-psnr}
    \begin{tabularx}{\linewidth}{XXXX}
      \toprule[1.5pt]
      $Bitrate_{cond}$(kb/s) & $PSNR_{REF}$(dB) & $PSNR_{PRO}$(dB) & $\Delta PSNR$(dB) \\
      \midrule[1pt]
      200 & 34.576 & 34.857 & 0.28  \\
      250 & 35.573 & 35.685 & 0.11 \\
      300 & 36.420 & 36.541 & 0.12  \\
      350 & 37.135 & 37.238 & 0.10 \\
      400 & 37.755 & 37.922 & 0.17  \\
      450 & 38.305 & 38.494 & 0.19 \\
      500 & 38.861 & 38.996 & 0.14  \\
      \bottomrule[1.5pt]
    \end{tabularx}
\end{table}

\begin{table}[H]
  \centering
    \caption{子节点加权和剪枝算法park\_joy实验(编码时间)}
    \label{tab:result-weighted-park-time}
    \begin{tabularx}{\linewidth}{XXXX}
      \toprule[1.5pt]
      $Bitrate_{cond}$(kb/s) & $Time_{REF}$(ms) & $Time_{PRO}$(ms) & $\Delta Time$(\%) \\
      \midrule[1pt]
      200 & 492815 & 414570 & -15.87  \\
      250 & 545196 & 392822 & -27.94 \\
      300 & 541272 & 425908 & -21.31  \\
      350 & 560691 & 431319 & -23.07 \\
      400 & 573799 & 456575 & -20.43  \\
      450 & 540554 & 462260 & -14.48 \\
      500 & 597913 & 474520 & -20.64  \\
      \bottomrule[1.5pt]
    \end{tabularx}
\end{table}

同样可以看到PSNR与比特率变化相差无几,而编码时间进一步减少。在miss\_am视频上的实验结果如
表~\ref{tab:result-weighted-miss-time} 所示,仅给出编码时间上的对比。

% TODO: miss_am_qcif结果
\begin{table}[H]
  \centering
    \caption{子节点加权和剪枝算法miss\_am\_qcif实验(编码时间)}
    \label{tab:result-weighted-miss-time}
    \begin{tabularx}{\linewidth}{XXXX}
      \toprule[1.5pt]
      $Bitrate_{cond}$(kb/s) & $Time_{REF}$(ms) & $Time_{PRO}$(ms) & $\Delta Time$(\%) \\
      \midrule[1pt]
      200 & 35468689 & 27476545 & -22.53 \\
      250 & 37811001 & 28443458 & -24.77 \\
      300 & 39371983 & 32556881 & -17.31 \\
      350 & 44189312 & 38573681 & -12.71\\
      400 & 49631606 & 38613336 & -22.20\\
      \bottomrule[1.5pt]
    \end{tabularx}
\end{table}

% avg: -19.9050710753

% TODO: 小结加权和算法成果
可以看到相较于基于子节点和剪枝的算法,使用加权和能进一步降低视频编码所用的时间。一点有趣的现象是,
在体积更大的视频miss\_am上,基于子节点加权和剪枝的早终止算法仍表现出了不错的效果,这可能
是由于AV1标准对于四种新的T型划分在后续编码流程中的处理较为薄弱,对这四种T型划分的严格筛选可以
有效加速编码过程。

% TODO: 对比表格?

\section{阈值剪枝算法实验结果}

如 \ref{sec:algorithm-threshold-try}节 所分析,笔者首先选取了各模式最佳RDCost分布中位于80\%处的值作为Threshold
进行实验,实验结果上,PSNR与比特率几乎不变,这里不再列出;而所用编码时间上,时间缩短约10\%至20\%不等。这一实验结果
与基于子节点和的剪枝算法所得到编码时间缩减程度近似。

% (Threshold = 16037601)
\begin{table}[H]
  \centering
    \caption{阈值剪枝算法park\_joy实验(Threshold=p(80\%))}
    \label{tab:result-threshold-park-time}
    \begin{tabularx}{\linewidth}{XXXX}
      \toprule[1.5pt]
      $Bitrate_{cond}$(kb/s) & $Time_{REF}$(ms) & $Time_{PRO}$(ms) & $\Delta Time$(\%) \\
      \midrule[1pt]
      200 & 492815 & 437619 & -11.20  \\
      250 & 545196 & 432917 & -20.59 \\
      300 & 541272 & 441926 & -18.35  \\
      350 & 560691 & 475368 & -15.21 \\
      400 & 573799 & 461355 & -19.60  \\
      450 & 540554 & 487155 & -9.88 \\
      500 & 597913 & 512709 & -14.25  \\
      \bottomrule[1.5pt]
    \end{tabularx}
\end{table}

% more threshold
% 50: 349230
% 55: 421811
% 60: 518589
% 70: 787617
接下来笔者尝试了更多的阈值,分别选取了位于最佳RDCost分布中50\%、55\%、60\%、70\%、80\%的值作为阈值进行实验,
结果如 表~\ref{tab:result-more-threshold-park} 所示。可以看到选择过小的阈值会导致许多节点产生不好的剪枝,
其结果是使得编码所用的时间不降反升;而当阈值的选取较为保守时,则可以产生节省10\%左右编码时间的优化效果。

\begin{table}[H]
  \centering
    \caption{不同阈值在park\_joy上的实验结果}
    \label{tab:result-more-threshold-park}
    \begin{tabularx}{\linewidth}{XXXXXXX}
      \toprule[1.5pt]
      Threshold & Bitrate(b/s) & $\Delta Bitrate$ & PSNR(dB) & $\Delta PSNR$ & Time(ms) & $\Delta Time$ \\
      \midrule[1pt]
      p(50\%) & 296800 & -0.12\% & 34.695 & 0.12 & 613047 & 24.40\% \\
      p(55\%) & 298200 & 0.35\% & 34.663 & 0.09 & 643915 & 30.67\% \\
      p(60\%) & 298360 & 0.40\% & 34.668 & 0.09 & 448292 & -9.03\% \\
      p(70\%) & 298360 & 0.40\% & 34.668 & 0.09 & 457853 & -7.09\% \\
      p(80\%) & 296920 & -0.08\% & 34.709 & 0.13 & 437619 & -11.20\% \\
      \bottomrule[1.5pt]
    \end{tabularx}
\end{table}


将这些阈值应用在miss\_am视频上进行测试,结果如 表~\ref{tab:result-more-threshold-miss} 所示。

% base: 123834 b/s, 45.756, 35468689 ms,

\begin{table}[H]
  \centering
    \caption{不同阈值在miss\_am上的实验结果}
    \label{tab:result-more-threshold-miss}
    \begin{tabularx}{\linewidth}{XXXXXXX}
      \toprule[1.5pt]
      Threshold & Bitrate(b/s) & $\Delta Bitrate$ & PSNR(dB) & $\Delta PSNR$ & Time(ms) & $\Delta Time$ \\
      \midrule[1pt]
      p(50\%) & 123163 & -0.54\% & 45.746 & -0.01 & 37543458 & 5.84\% \\
      p(55\%) & 123495 & -0.27\% & 45.753 & -0.003 & 43106798 & 21.53\% \\
      p(60\%) & 123709 & -0.10\% & 45.752 & -0.004 & 43311828 & 22.11\% \\
      p(70\%) & 123391 & -0.36\% & 45.758  & 0.002 & 42662776 & 20.28\% \\
      \bottomrule[1.5pt]
    \end{tabularx}
\end{table}

该结果表明笔者所选取的基于park\_joy上的简单统计所得到的阈值均不适用与miss\_am视频。根据趋势分析,
适合该视频的阈值可能会更小。由这两个视频上的测试可以看出,选取一系列合适的阈值用于剪枝是非常困难的,
一种可行的思路应当是根据一些易于获取的视频特征,如分辨率、码率等,来确定阈值。